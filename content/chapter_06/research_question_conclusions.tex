\section{Conclusion on Research Questions}

Alterations to tempo, key and noise audio facets are made to music by remastering
as discussed in Chapter \ref{chapter:design}. Music with noise alterations can be
identified with \say{Radio Broadcast Monitoring to Ensure Copyright Ownership}\cite{Nishan}.
Therefore, identifying music with tempo and key alterations was crucial to identify remastered
music in radio broadcasts.

There are several approaches to identify music with tempo and key alterations as introduced in
Chapter \ref{chapter:lit_review}. Majority of the literature to identify music with tempo and
key alterations were \ac{stft} based. Using \ac{sift} descriptors to match two \ac{stft} spectrograms
suits identification of remastered songs as it can be used to match variable sized audio clips with 
original songs. Above 97\% accuracy can be achieved for tempo alterations up to 20\% and above 95\%
accuracy can be achieved for pitch alterations up to 20\% as discussed on Section \ref{section:results}.    
Hence, it can be concluded that research questions were successfully answered by accomplishing research
objectives. 
